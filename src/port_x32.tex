\begin{wrapfigure}[4]{r}{0.4\textwidth}{
\centering \scaledimage{0.4}{sega_x32_logo.png}}
\end{wrapfigure}
In January 1994, Sega was in a delicate position. The Genesis, its 16-bit moneymaker, was losing ground in Japan. By 1993 sales had placed the machine third, behind Nintendo's Super Famicom and NEC's PC Engine. To make things worse, Sega now had to deal with two new competitors who had entered the game console market in 1993, with Atari's Jaguar and Panasonic's 3DO. The consensus at Sega Of Japan (SOJ) was that the company should put all of its available resources into the 32-bit Saturn project.\\
\par
While SOJ's work on the Saturn was moving forward, it was feared it would be a while before it would be finished. In the US, the Genesis had been selling well (32 million units as of late 1993) and Sega of America (SOA) was eager to take the financial opportunity to create a Genesis "booster".\\
\par
During CES '94 in Las Vegas, then-CEO of SOJ Hayao Nakayama summoned Sega of America (SOA) executives Joe Miller -- Head of R\&D, Marty Franz -- SOA Technical Director, and Scot Bayless -- Senior Producer, into a conference call\footnote{Source: Retrogamer \#77. All quotes in this section are also from Retrogamer interviews.}.\\
\par 
They were given the green light for project "Mars" with the goal to release a Genesis Booster within nine months. Incredibly, they managed to reach their target. The Sega 32X was released in November 1994.\\
\par
\fullimage{consoles/32X.png}%{Sega Genesis (a.ka. MegaDrive) with a 32X booster on top of it.}
\par
%The 32X is inserted into the Genesis like a standard game cartridge.
%\pagebreak
The 32X, as it would be marketed, was to be inserted like a cartridge. 32X games were inserted on top of the overall assembly. Games had access to everything including the Genesis's 7.6 MHz Motorola 68000 and the 3.58 MHz Zilog Z80.\\

\fq{After the call ended, Marty Franz grabbed one of those little hotel notepads and drew a couple of Hitachi SH2 processors, each with its own frame buffer. That's pretty much where the 32X started.}{Scot Bayless}\\
\par

\drawing{x32_arch}{What the notepad may have looked like.}

\par
\fq{The design of the graphics subsystem was brilliantly simple; something of
a coder's dream for the day. It was built around two central processors feeding independent frame buffers with twice the depth per pixel of anything else out there. It was a wonderful platform for doing 3D in ways that nobody else was attempting outside the workstation market.}{Scot Bayless}\\
\par

Besides the dual SH-2, the 32X was gifted an impressive audio chip from QSound. Capable of Pulse-Width Modulation, it added extra channels and even had multidimensional sound capability that allowed a regular stereo audio signal to approximate the 3D sounds heard in everyday life. Also present was a graphics chip named "VDP" that was in charge of double-buffering to avoid tearing and was also capable of clearing the framebuffers rapidly.
\par

\pngdrawing{x32_fullarch}{The 32X system as summarized in the developer documentation.}

The design bore similarities to the Saturn (which also used two SuperH CPUs) but with a different philosophy.\\
\par
\fq{The Saturn was essentially a 2D system with the ability to move the
four corners of a sprite in a way that could simulate projection in 3D space, It had the advantage of doing the rendering in hardware, but the rendering scheme also tended to create a lot of problems, and the pixel overwrite rate was very high; much of the advantage of dedicated hardware was lost to memory access stalls. The 32X, on the other hand, did everything in software but gave two fast RISC chips tied to great big frame buffers and complete control to the programmer.}{Scot Bayless} \\
\par
With engineers pouring their hearts into the system and DevRel doing amazing work to have a decent portfolio of games for launch date, the 32X managed to sell 665,000 units by the end of 1994. This promising start was unfortunately followed by a sad story.

A story best summarized by Damien McFerran, reporter at RetroGamer.\\
\par
\fq{How do you take half a decade's worth
of critical and commercial success and  flush it down the toilet?\\
\par
 Easy: you release a device like the Sega 32X.}{Reporter for RetroGamer \#77}\\
\par

What crippled the 32X was the Saturn. Throughout 1994, work had continued at SOJ on the 32-bit system. Enough progress was made that Sega decided to release it in Japan in November 1994, way ahead of its original schedule. This was the same month the 32X was to be released in USA.\\

\par

\fq{Not surprisingly, word got out quickly in the West, US and EU consumers immediately started asking the obvious question: 'Why should I buy 32X when Saturn is only a few months away?' Sadly, the best answer Sega could come up with was that 32X was a 'transitional device' - that it would form a bridge from Mega Drive to Saturn.\\
\par
 It made us look greedy and dumb to consumers, something that a year earlier I couldn't have imagined people thinking about us. We were the cool kids.}{Scot Bayless}\\
 \par This poor timing made the 32X almost dead on arrival. Not only was the Saturn just around the corner, the  PlayStation was released one month after the 32X on December 3, 1994. By the end of 1995, inventory of the 32X was sadly liquidated at \$19.95 per unit.\\
\par
Looking back on this era and reading interviews gives a bitter feeling when you keep in mind that, up until that point in history, Sega had been a colossal competitor to Nintendo. It had a cool image which had taken five years to build\footnote{During 1993, Sega was the biggest advertiser on MTV (source: "RetroGamer \#77").}.\\
\par
 From this point it looks like the company made one bad decision after another. Sega's final console, the Dreamcast released in 1998, ended up being widely popular but sales were not enough to save the hardware business. Sega abandoned the market to focus on programming games instead.





Looking back over his time trying to ship the 32X on schedule, Scot Bayless provided insightful memories.\\
\par
\fq{32X games in the queue were effectively jammed into a box as fast as possible, which meant massive cutting of corners in every conceivable way. Even from the outset, designs of those games were deliberately conservative because of the time crunch. By the time they shipped they were even more conservative; they did nothing to show off what the hardware was capable of.}{Scot Bayless}\\
\par
Deeper issues rooted in Sega, Inc.'s culture seem to explain later mistakes.\\
\par
\fq{
The 32X is a great case study in two things:\\
\par
First, messaging: your number one job in marketing
is to establish the value proposition. Even with all the rushed hardware and late software, if Sega had been able
to convince people that the 32X was really worth having, it might have had
a chance to succeed. But we never did that; we never managed to explain to anyone in any credible way what was so unique and worthy about the 32X. The result is exactly what you'd expect: Sony ate our lunch.\\
\par
 Second: honesty; not in the legal sense, nor in the public sense, but internally. I remember when I arrived at Microsoft in 1998 I attended an executive orientation briefing on my first day. The VP who met with us said: 'The one thing we demand of every one of you guys is to say what you think.' That attitude was what kept Microsoft vibrant, healthy and successful for
more than 20 years.
 Sega, by contrast, lacked that ruthless honesty. Nobody wanted to hurt anyone's feelings. Even when everybody knew the 32X and Saturn were way behind the power curve, nobody was willing to stand
up and say so. And it wasn't just the hardware; during the same period, Sega published some of the oddest games it ever released. Games that were deeply flawed. Games that completely failed
to connect. And all the while everyone was smiling and saying, 'Gosh, aren't we great?'' I wasn't able to articulate all this at the time, but I know I felt it intuitively. I knew there was something wrong, that we were losing our way.
}{Scot Bayless}





\subsection{\doom{} On 32X}
If porting \doom{} to Jaguar had been a tour-de-force, repeating the feat on a system even less powerful was to require nothing short of a miracle. Once again John Carmack invested himself totally in the project.\\
\par
\fq{I spent weeks working with Id Software's John Carmack, who literally camped out at the Sega of America building in Redwood City trying to get Doom ported. That guy worked his ass off and he still had to cut a third of the levels to get it done in time.\\ 
\par
What amazes me now is that with all that going on, nobody at Sega was willing to say "Wait a minute, what are we doing? Why don't we just stop?" Sega should have killed the 32X in the spring of 1994, but we didn't. We stormed the hill, and when we got to the top we realized it was the wrong damn hill.\\
\par
Looking back now I'd say that really was the beginning of the end for Sega's credibility as a hardware company.}{Scot Bayless}\\
\par
To make the game fit in the 512KiB RAM of the 32X, even more features than the Jaguar version had to be cut. Another enemy, the Spectre, had to be removed. Additional poses of monsters were removed except for the ones facing the player. Since they could no longer face each other, monster infighting was also removed. There were no savegames; players instead manually selected the starting level instead. The cartridge only had enough room for seventeen heavily-edited maps. Since none of them had the BFG9000, the weapon was unavailable (but could be obtained using cheat codes).\\
\par
There was also a significant problem of performance. Even with its twin SuperHs, the machine was unable to render at the original resolution.\\
\par
\fq{I liked the 32X -- it was basically two decent 32-bit processors (SH2) and a framebuffer, so you programmed like on a PC, but with SMP long before it was mainstream on PC.  It was still pretty underpowered compared to even a 386, so resolution was low.}{John Carmack}\\
\par
\cfullimage{consoles/x32_screenshot.png}{E1M1's legendary entrance hall}

In figure \ref{consoles/x32_screenshot.png} notice how, like on the Jaguar, E1M1's blue floor texture had to be replaced with a brown one to limit RAM consumption. Likewise, the number of steps on the stairs was also lowered in order to reduce the number of visplanes generated.\\
\par
The game ran at a resolution of 320x224 but the CPUs struggled so much that the active window was reduced to 128x144 (column-doubled to reach 256x144), leaving 100 vertical pixels for the status bar and the brown border/background. With all these compromises the framerate managed to reach the 15-20 FPS\footnote{Source: Digital Foundry YouTube channel, "DF Retro: Doom - Every Console Port Tested and Analysed!".} range, which gave a pleasant experience.\\
\par
\trivia{Sega named all its projects after planets of the solar system. Besides Saturn and Mars, two others are known. Neptune was a two-in-one Genesis and 32X console that Sega planned to release in the fall of 1995. It was canceled because of fears that it would dilute their marketing for the Saturn while being priced too close to the Saturn to be a viable competitor. Jupiter is rumored to have been a Saturn without a CD drive.}



\fullimage{32xe1m1.png}\\
\par
Map complexity was heavily reduced. Above, E1M1's main room was stripped of many textures (compare to the PC version on page \pageref{complex_scene_plain_light.png}). Below, the "pit" of E1M3 was flattened.\\
\par
\fullimage{32xe1m2.png}






