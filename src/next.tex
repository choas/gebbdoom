\section{History}

\begin{wrapfigure}[8]{r}{0.20\textwidth}
\centering
\includegraphics[width=.20\textwidth]{drawings/NeXT_logo.pdf}
\end{wrapfigure}
\par
NeXT's history starts (and amusingly, also ends) at Apple. In May of 1985, the mediocre sales of the Macintosh painted a bleak future for the company. Steve Jobs, co-founder and then General Manager of the Mac department, wanted to lower the price and increase marketing in order to boost the Mac. John Sculley then CEO, wanted to abandon the Mac and refocus the company's resources on the Apple II, the only profitable product Apple had marketed until then.\\
\par
 A vote was called and the board of directors sided with Sculley. Steve Jobs found himself stripped of all responsibilities. A few months later, on September 13, 1985, he resigned and went on to work on his next project.\\
\par
NeXT, Inc. was incorporated in February 1986 with \$7 million of Jobs' own money. Many members of the Mac division left Apple to join the newly formed company, among them  Joanna Hoffman, Guy "Bud" Tribble (head of software division), George Crow, Rich Page, Susan Barnes, Susan Kare, and Dan'l Lewin.\\ 
\par
With NeXT, Jobs went back to a project he had contemplated for Apple in August 1985. While touring universities to boost Mac sales, he had met Paul Berg, a Nobel Laureate in chemistry. Paul was frustrated with the cost\footnote{\$100,000.} of teaching students about recombinant DNA in wet laboratories. It would have been cheaper to simulate them. It seemed there was a market for 3M\footnote{One Megabyte of RAM, a Megapixel display and MegaFLOP performance.} workstations targeted at universities and students\footnote{The Second Coming of Steve Jobs.}. NeXT set itself to build something powerful yet cheap enough that college students could afford it.\\
\par
\fq{I want some kid at Stanford to be able to cure cancer in his dorm room.}{Steve Jobs, 1987}\\
\par
Steve Jobs spared no expenses. For \$100,000, Paul Rand was commissioned with a logo. An automated factory featuring automated surface-mount motherboard assembly\footnote{Source: "The Machine to Build The Machines" mini documentary.} capable of producing 10,000 units per month was built in Fremont, CA. The design firm Frogdesign, which had proven itself with the Apple IIc, was hired. The goal was to ship by the end of 1986.\\
\par
The machine was to be perfect, following Alan Kay's concept of creating both the hardware and the software to run it.\\
\par
\fq{People who are really serious about software should make their own hardware.}{Alan Kay, 1980}\\
\par

Using their experience from Apple and particularly their work on the Macintosh, the company defined the three pillars of the NeXT Computer: GUI, Networking and Object-Oriented programming.\\
\par
\fq{I went to Xerox PARC. And they were very kind. They showed me what they are working on. And they showed me really three things. But I was so blinded by the first one that I didn't even really see the other two. One of the things they showed me was object oriented programming -- they showed me that but I didn't even see that. The other one they showed me was a networked computer system... they had over a hundred Alto computers all networked using email etc., etc., I didn't even see that. I was so blinded by the first thing they showed me, which was the graphical user interface. I thought it was the best thing I'd ever seen in my life. }{Steve Jobs, 1995}\\






\par
\section{The NeXT Computer}
The first machine shipped in 1989 after three years of hard work. Not meeting the initial release target was not a problem according to Jobs who famously replied to a journalist inquiring about the delay: "Late? This computer is five years ahead of its time!".\\
\par
Based on a Motorola 68030 25 MHz with 8 MiB RAM and featuring powerful co-processors such as a DSP and a FPU, the high-performance hardware delivered. The machine also happened to be gorgeous. In an era where most computer cases were made of beige plastic, the elegance of the one foot perfect cube made of painted magnesium stood out.\\
\par
\cfullimage{NEXT_Cube}{The Next Computer}


The monitor was a piece of art itself. The 17" MegaDisplay allowed a high\footnote{At the time, a 14" monitor delivering a resolution of 640x480 was high-end standard on PC.} resolution of 1120 x 832 pixels with a density of 92 DPI. The Cube's 256 KiB of VRAM allowed four shades of gray per pixel. At launch the supply chain was so tight that when ordering a NeXT Computer, the customer received two parcels -- one from Fremont containing the central unit, and another directly from Sony containing the MegaDisplay.





\begin{figure}[H]
\centering
\cscaledimage{1}{68030_blueprint.png}{Motorola 68030}
\end{figure}
\par
One of the many innovations of the NeXT Computer was its reliance on the 256 MiB magneto-optical drive, a hybrid between a HDD and floppy disk aimed at filling both use cases. According to Steve Jobs, it was supposed to allow users to "take their whole world in their backpacks".\\
% Released in 1987
\par
At the heart of the machine, the 32-bit 68030 was the latest in Motorola's 68000 series. The choice was likely influenced by the experience NeXT hardware engineers had built while working on Apple's Macintosh and Lisa (both were powered by a 68000).\\
\par
 Running at a frequency of 25MHz, it was able to execute nearly 5 MIPS. It did not feature a built-in FPU, so a Motorola 68882 was placed next to the CPU on the motherboard. %\footnote{Of course, Motorola documentation claimed almost double that with 12 Mips!}



\begin{figure}[H]
\centering
\cscaledimage{1}{68030_layout.png}{Motorola 68030 diagram\protect\footnotemark }
\end{figure}
\footnotetext{Source: "The NeXT Book" by Bruce F. Webster.}
\par
\vspace{-3mm}
Above, the 273,000 transistors of the 68030, made up of \circled{1} Memory Management Unit, \circled{2} $\mu$ROM, \circled{3} nROM, \circled{4} Control Section, \circled{5} Instruction Pipe, \circled{6} Program Counter Execution Unit, \circled{7} Address Execution Unit, \circled{8} Data Execution Unit, \circled{9} 256 bytes i-cache, \circled{A} 256 bytes d-cache, and \circled{B} Clock Generator.\\         
\par
It is unclear how much of a performance boost the two caches provided. Their small size of 256 bytes each would have meant a significant cache miss rate (Intel had discarded its on-die cache from their 386 for this very reason). Interestingly, the designer decided to use both micro-code and nano-code. Sixteen general-purpose registers were available which was atypical for a CISC architectures\footnote{Intel's CISC-based 486 CPU had eight.} (that many registers were usually found in RISC).
 


\begin{wrapfigure}[9]{r}{0.33\textwidth}
\centering
\scaledimage{0.33}{next/next-cube-system.png}
\end{wrapfigure}


Contrary to PCs which were a mess of wires, the NeXT Computers formed a chain. The mouse was connected to the keyboard, itself connected to the screen, connected to the Cube.\\


\par


If initially the NeXT Computer was acclaimed for its specs, there was a serious issue with the price. Market studies showed that students and researchers wanted a workstation priced at \$3,000. The NeXT Computer started at more than twice the ideal price at \$6,500. To make it worse, the optical drive that powered the basic configuration would turn out to be great for backup but way too slow for runtime. Not only was it noisy and unreliable, it offered an access time of 90 ms, 10 times slower than a hard-drive and made the operating system crawl. This rendered the "optional" \$3,500 330 MiB SCSI hard-drive an absolute necessity, pushing the final price tag to \$10,000! A big price to pay for a machine not even able to output color.\\
\par




\section{Line of Products}
Given the low sales of the NeXT Computer, the original machine was discontinued and the line of products refreshed. In 1991 NeXT released three new products\footnote{Announced four months in advance on September 18, 1990.}. The NeXTcube was the direct successor to the NeXT Computer. A smaller, flattened version of the NeXTcube called the NeXTstation offered built-in color capability but no expansion slots. Last but not least, there was a graphic and video processor expansion board called NeXTdimension.\\
\par
\newcolumntype{L}[1]{>{\hsize=#1\hsize\raggedright\arraybackslash}X}%
\newcolumntype{R}[1]{>{\hsize=#1\hsize\raggedleft\arraybackslash}X}%
 \begin{figure}[H]
\centering  
\begin{tabularx}{\textwidth}{ L{1.7}  R{0.3}  R{1.4}  R{0.8}  R{0.8}}
  \toprule
  \textbf{Name} &  \textbf{Year} & \textbf{CPU} & \textbf{Price} & \textbf{in 2018}   \\
  \toprule 
   NeXT Computer           & 1989 & 68030 25 MHz & \$6,500 & \$12,938 \\
\toprule 
   NeXTstation             & 1991 & 68040 25 MHz & \$4,995 & \$9,157 \\
   NeXTcube                & 1991 & 68040 25 MHz & \$12 395 & \$21,171 \\
   NeXTdimension           & 1991 & i860  33 MHz & \$3,995 & \$7,552 \\
   NeXTstation Color       & 1991 & 68040 25 MHz & \$7,995 & \$14,656 \\
\toprule 
   NeXTcube Turbo          & 1992 & 68040 33 MHz & \$10,000 & \$18,121 \\
   NeXTstation Turbo       & 1992 & 68040 33 MHz & \$5995 & \$11,932 \\
   
   NeXTstation TurboColor & 1992 & 68040 33 MHz & \$8995 & \$17,904 \\
   \toprule
\end{tabularx}
\caption{\protect\NeXT products from 1989 to 1993\protect\footnotemark.}
\end{figure}
\par
\footnotetext{Source: kevra.org (Competing Hardware Comparisons), https://simson.net/ref/NeXT/specifications.htm, and "The Second Coming of Steve Jobs".}
\par
In 1992, they buffed up their entire line with Turbo versions and what would become the Gold Standard at id Software: The NeXTstation TurboColor.






% The only weakness of the standard and turbo versions was the display.
% The standard and turbo versions' only weakness remained the video.

\section{NeXTcube}
From the outside the NeXTcube's 12" cubic central unit looked exactly like its predecessor the NeXT Computer. However, the inside told a different story.\\
\par
 The CPU was bumped to a Motorola 68040 25MHz, a chip capable of three times the 68030's throughput with 15 MIPS\footnote{Source: "Fast New Systems from NeXT", B.Y.T.E Nov 1990}. The machine's RAM capacity was doubled, with an out of factory 16 MiB, expandable to 64 MiB. The magneto-optical disc was abandoned in favor of a mandatory HDD, floppy disk reader, and an optional CD-ROM drive. The HDD capacity was augmented with a choice of 400 MiB, 1.4GiB or 2.8GiB SCSI drive. The floppy disks were twice the capacity of PCs at 2.88 MiB.\\
\par
\begin{minipage}{\textwidth}
\scaledimage{0.5}{next/next-crt-top.png} \scaledimage{0.5}{next/next-cube-top.png}\\
\end{minipage}

\par
The NeXTcube Turbo released in 1992 was almost the same machine, except the 68040's frequency was bumped to 33 MHz and the max RAM capacity increased to 128 MiB.\\
\par
The only weakness of the standard and turbo versions was the display. Shipping with 256 KiB of VRAM, the machine could only output four colors (white, black and two shades of gray). To bring color to the NeXTcube, customers had to invest in a NeXTdimension board.\\
\par

\trivia{The NeXTcube Turbo expansion slot could welcome a Nitro board that replaced the 68040 33MHz with a 68040 40 MHz. Only 10 Nitro boards are known to exist, they are extremely rare and highly sought after by collectors.}
\pagebreak


\cfullimage{next/NeXTcube_motherboard.png}{NeXTcube motherboard}
\par
Opening the machine revealed the minutiae NeXT had adopted as its standard. The NeXTcube motherboard above shows that aesthetics were not sacrificed for performance. Surface mounting allowed components to be placed much closer to others than usual.\\
\par
Hidden under a heat-sink\footnote{Heat dissipation was always a problem for the 68040 which prevented running at high frequencies, a handicap against Intel's 486 capable of 66 MHz.} and packing 1.2 million transistors, the CPU was a big step up. The Harvard architecture (separated storage and signal pathways for instructions and data), write-back capability, 8 KiB cache (4 KiB data and 4 KiB instructions), and integrated FPU tripled throughput compared to the 68030.





\vspace{50 mm}
\drawing{NeXTCube_motherboard}{NeXTcube motherboard diagram}
\par
Chipsets and components of the NeXTcube motherboard:\\
\par 
\circled{1} NeXTBus connector,
\circled{2} VLSI NeXTBus Interface Chip,
\circled{3} CPU Motorola 68040,
\circled{4} 256~KiB~VRAM,
\circled{5} DRAM Controller CS38PG017CG01,
\circled{6} Integrated Channel Processor (DMA Controller Fujitsu MB610313),
\circled{7} Optical Storage Processor (Fujitsu MB600310),
\circled{8} 16 SIMM slots max 4MiB each for total 64 MiB,
\circled{9} DSP-56001RC20,
\circled{A} Battery,
\circled{B} NeXT BIOS PROM,
\circled{C} DSP 768K Slot,
\circled{D} Hard-Drive and Floppy connectors.
\circled{E} Many connectors (top to bottom): 56001 DSP, Serial Port A\&B, SCSI2, Printer, Ethernet RJ45\&CoaxBNC, DB19 Monitor. 
\circled{F} Intel n82077 Disk controller.
\circled{G} DSP SRAM (8KiB) MCM56824A.
\circled{H} SCSI Controller (NCR 53C90A)
\circled{I} 100.000MHz Oscillator K1149AA






\section{NeXTstation}

Since cost was the main issue with their product line, NeXT attempted to introduce a less expensive product. They designed something close to the NeXTcube but removed non-essential elements in order to produce a three times cheaper, all-in-one machine.\\
\par
The NeXTstation's pricing and appearance made it a direct competitor to the SPARCstation. No longer a perfect cube, the case, nicknamed "the slab" (and also, banned by Steve Jobs, the "pizza box") was well-received by customers and became \NeXTns{}'s most successful computer.\\
\par 
\cfullimage{cube_vs_slab.png}{A NeXTstation ad from NeXTWorld 1991 magazine.}
\par
\vspace{-10pt}
Designers picked elements from the NeXTcube and the NeXTdimension in order to produce an all-in-one, non-extensible machine. The three NeXTBus expansion slots were removed and so was the CD-ROM. A 2.88MiB floppy disk was added on the right side. The most notable difference came in the color version that had 2 MiB of VRAM, making the machine capable of 16-bit RGB colors. To accommodate the increased bandwidth requirements, the motherboard was redesigned to include a Bt463 RAMDAC.\\
\par
% The system also introduced usage of the Apple Desktop Bus created by Steve Wozniak which bear many similarities with the USB released in 1996. With ADB, input devices such as mouse and keyboard are connected as a daisy chain\footnote{ADB - The Untold Story: Space Aliens Ate My Mouse.} and can be plugged/unplugged safely while the machine was powered on without fearing to fry something.\\
\fq{The 16-bit color was only 4444, 1555 was not supported, which was unfortunate for us.  It was also in a linear color space\footnotemark, as opposed to the non-linear PC standard that became sRGB.  I didn't understand how to properly convert back then, so our graphics always looked washed out on the NeXT systems.}{John Carmack}\footnotetext{NeXT, SGI, and Apple had linear color space. PC of course did not.}



\cfullimage{next/nextstation.png}{A NeXTstation TurboColor (non-ADB)}
\par
\vspace{-5pt}
In an unmistakable Steve Jobsian fashion, several components were renamed to emphasize subtle differences. The central unit fan was called a "whisper fan"\footnote{Source: "Fast New Systems from NeXT", B.Y.T.E Nov 1990}.  The power supply was a 120-watt unit using a new technology called "parallel resonant switching" that allegedly allowed a much smaller form factor than conventional power supplies.\\
\par
Despite its reduced size, the NeXTstation's performance didn't suffer compared to the more imposing NeXTcube. The four variants -- NeXTstation, NeXTstation Color, NeXTstation Turbo, and NeXTstation TurboColor -- all relied on a 68040 with 12 MiB of RAM.\\
\par
\trivia{No button or switch are visible on the computer itself. The machine could only be turned on and off via the keyboard (a novelty at the time). An update later introduced the Apple Desktop Bus created by Steve Wozniak which bears many similarities to the USB standard released in 1996.}
\pagebreak









\section{NeXTdimension}
\vspace{-3pt}
The 256 KiB of VRAM on the NeXTcube only allowed a mediocre four shades of gray. The NeXTdimension was to take the workstation to a whole new level. Shipping with 4 MiB of VRAM and 8MiB of RAM (extensible to 32 MiB), it allowed a 24-bit color per pixel GUI and real-time recording/playback of video signals. Since the board was connected via a NeXTBus port, up to three NeXTdimensions could be connected, allowing the NeXTcube to drive four extended screens simultaneously.\\
\vspace{-2pt}
\par
On presentation day, Steve Jobs managed to demonstrate the groundbreaking capabilities in his signature spectacular fashion. A sequence from the black-and-white movie Alice in Wonderland was played live on a NeXTcube, already a tour-de-force at the time. The audience was impressed yet the best was to come. As Alice progressed through Wonderland, frames progressively turned to color. The audience went berserk.\\
\par
At some point the NeXTdimension was even planned to feature real-time video compression, but problems prevented it.\\
\par
\fq{NeXT has eliminated the C-Cube Microsystems CL-550 JPEG chip from NeXTdimension. This is because our supplier, C-Cube Microsystems, has failed to deliver chips that meet their specifications.}{Felipe\_Fuster\at NeXT.COM}\\
\par
\vspace{-2pt}
The NeXTdimension was not a mere expansion board, but rather a full-featured computer within the computer. It had its own operating system, RAM, and clock generator which communicated with NeXTSTEP via Mach messages.\\
\par
\fq{The NeXTdimension ran a custom kernel which was designed to do soft real-time management of multiple threads within a single address space, provide demand paged virtual memory, and provide a source-compatible Mach API subset and full Mach messaging interface, along with a minimal UNIX system call API, just enough to implement the RenderMan back end and the PostScript device layer. The kernel was called "Graphics aCcelerator Kernel, or "GaCK". Yes, this was a jape at the funny capitalization of the company name. It was not Mach, or BSD, or Minix, or Linux.}{M Paquette, NeXT Engineer}\\
\par
\vspace{-5pt}
The card came with the \cw{NeXTtv.app} which allowed video editing and frame capture.
\par
\vspace{3pt}
\fullimage{NeXTtv.png}\\
\par
Because of Steve Jobs's "hobby" venture with Pixar, the NeXTdimension had close ties with RenderMan. It had a built-in hardware acceleration module called Quick RenderMan.\\
\par
% Renderman uses a similar architecture. The 3DKit and RI C binding reside on the m68k (otherwise you couldn't link with them). 
\fq{
Depending on the setup of the Renderman context, a RIB stream can be spooled to Photorealistic Renderman running on the host CPU (the m68K for black hardware), or to a Quick Renderman implementation loaded on demand into the Window Server. The Quick Renderman implementation in the Window Server may then, if the target window is on a NeXTdimension, forward the rendering operations to a Quick Renderman context running on the NeXTdimension board.}{M Paquette, NeXT Engineer} 
\par






\cfullimage{next/NextDimension_board.png}{NeXTdimension board}
\par
Like the NeXTcube and NeXTstation motherboards, the NeXTdimension hardware was gorgeous and benefited from the same "surface mount" manufacturing process. The most prominent component is of course the Intel 860.\\
\par It had failed to beat Intel's 486 CPU in the market but its eagerness to participate in the \doom{} phenomenon allowed it to land a gig on the tools team.\\
\par
\trivia{Notice the Bt463 RAMDAC which would also be used on NeXTstations.}
\par
\drawing{NeXTDimension_motherboard}{NeXTdimension board diagram}
\par
List of chipsets and components of the NeXTdimension motherboard:\\
\par 
\circled{1} NeXTBus connector,
\circled{2} VLSI NeXTBus Interface Chip (NBIC),
\circled{3} Intel i860 CPU 33MHz 64-bit RISC CPU,
\circled{4} Motorola U88 Memory Controller,
\circled{5} Bt463 Ramdac,
\circled{6} Motorola U52 Data Formatter,
\circled{7} 4 MiB VRAM,
\circled{8} SiMM RAM extension slots up to 8x4= 32MiB,
\circled{9} Video color space conversion and video input (SAA 7191 WP \& SAA 7192 WP), 
\circled{A} Many connectors (top to bottom): Video Out(EGA/VGA, S-Video, Composite), Video In(S-Video, Composite, Composite), DB19 Monitor,
\circled{B} 33.000MHz Oscillator K1100AA,
\circled{C} 100.000MHz Oscillator K1149AA
\par

\cfullimage{NeXTDimension.png}{An early NeXTdimension ad from NeXT Magazine 1991.}
\par
Notice the C-Cube Microsystems JPEG chip which had to be cut from the final design, and the different resin, resulting in a different color for the motherboard. No doubt the color tone was the subject of many debates at NeXT headquarters. The "unlucky forever" Intel i860 was mislabeled "1860".
\pagebreak


To select the i860 as the main chip for a graphics card could have been called overkill but it was a careful and ultimately sound decision. Video processing is an intensive CPU task. It benefits so much from the i860's SIMD pipeline that nothing else but Intel's "Cray on a Chip" could have done the job.\\
\par
\fq{
The Intel 80860 was an impressive chip, able at top speed to perform close to 66 MFLOPS at 33 MHz in real applications, compared to a more typical 5 or 10 MFLOPS for other CPUs of the time. Much of this was marketing hype, and it never become popular, lagging behind most newer CPUs and Digital Signal Processors in performance.
The 860 has several modes, from regular scaler mode to a super-scalar mode that executes two instructions per cycle and a user visible pipeline mode (instructions using the result register of a multi-cycle op would take the current value instead of stalling and waiting for the result). It can use the 8K data cache in a limited way as a small vector register (like those in supercomputers). The unusual cache uses virtual addresses, instead of physical, so the cache has to be flushed any time the page tables changes, even if the data is unchanged. Instruction and data buses are separate, with 4G of memory, using segments. It also includes a Memory Management Unit for virtual storage.\\
\par
The 860 has thirty two 32 bit registers and thirty two 32 bit (or sixteen 64 bit) floating point registers. It was one of the first microprocessors to contain not only an FPU as well as an integer ALU, and also included a 3-D graphics unit (attached to the FPU) that supports lines drawing, Gouraud shading, Z-buffering for hidden line removal, and operations in conjunction with the FPU. It was also the first able to do an integer operation, and a (unique at the time) multiply and add floating point instruction, for the equivalent of three instructions, at the same time.\\
\par
However actually getting the chip at top speed usually requires using assembly language - using standard compilers gives it a speed closer to other processors. Because of this, it was used as a co-processor, either for graphics, or floating point acceleration, like add in parallel units for workstations. Another problem with using the Intel 860 as a general purpose CPU is the difficulty handling interrupts. It is extensively pipelined, having as many as four pipes operating at once, and when an interrupt occurs, the pipes can spill and lose data unless complex code is used to clean up. Delays range from 62 cycles (best case) to 50 microseconds (almost 2000 cycles).}{John Bayko's "Great Microprocessors of the Past and Present"}
\pagebreak
















\section{NeXTSTEP}
  NeXT's 1990 24-pages brochure, "Welcome to the NeXT decade" introducing the NeXT Computer System, laid out the seven pillars of the system they were about to build.\\
\par
\rawfq{
 Our collaboration with Higher Education provided the insight needed to visualize the seven breakthroughs that would ultimately define the NeXT Computer:\\
 \par
\begin{enumerate}
\item A new architecture optimized for total system throughput, not just individual component benchmarks.
\item A pioneering technology for vast and reliable storage, opening the door for new ways to access and use information.
\item Built-in CD-quality sound, allowing sound to be integrated into applications that are used every day.
\item A unified imaging system - Display PostScript - for both the display and the printer. So what you see on the screen is unequivocally what you get on paper.
\item An intuitive interface that gives everyone access to UNIX, with all of its power for networking and multitasking.
\item A multimedia mail system that enables communication combining text, graphics, and voice.
\item A new development environment that dramatically cuts the time it takes to create and customize software.
\end{enumerate}
~
}\\
\par


The first three points were the hardware team's responsibility. Everything else was on the software team and the amount of work ahead of them was overwhelming. The magical operating system described did not exist. To deliver their vision, they had to build it.\\
\par
 To create an OS from scratch would have been a humongous effort. To save time, the software team decided to reuse available components. They selected a microkernel called Mach from Carnegie Mellon University and combined it with elements from BSD (from University of California, Berkeley), such as the network stack, multi-user, multi-processing and filesystem. That was enough to bring the machine up to a command prompt.
\par
\subsection{GUI}
 To achieve the "unified imaging system", \NeXT engineers started from PostScript -- the language designed by Adobe for high-end printers -- and modified it to meet the need of a GUI in terms of look-and-feel and performance. The result was called Display PostScript\footnote{When OpenSTEP was used to build Rhapsody at Apple, the display system was changed to Portable Document Format (PDF) imaging model.}.\\
\par
\cfullimage{next/nextstep.png}{NeXTSTEP 1.0 running on a NeXT Computer in four color mode.}
\par
The resulting graphical system had many impressive features.\\
\par
 Some were raw power accomplishments, like for example the ability to see the content of a window while moving it (in all competing GUI-based OSes, windows had to be moved with only the outline visible because of graphic card and bandwidth limitations). Others were purely based on superior design and so solid that they remain in one form or another all the way up to Mac OS X running on 2018 Apple computers. The desktop metaphor, the unified titlebar scheme, the Dock, the bundles, and File Manager column view flow are only the most famous in a long list of GUI innovations.




\fq{The Display PostScript system can be broken into two pieces, the PostScript interpreter and the device. The interpreter processes the language, and passes marking, imaging, and compositing directions to the device layer.\\ 
\par
The device layer takes the high level marking, imaging, and compositing operations and (eventually) converts these to bitmap level operations. The Display PostScript system spends the majority of its time down here. In the case of the NeXTdimension board, the device layer is implemented on the NeXTdimension board. Marking, imaging, and compositing operations are asynchronously transmitted to the NeXTdimension for processing while additional PostScript is interpreted on the 68K processor. A good degree of parallelism is achieved in normal operation.}{M Paquette} 
%The NXPing() AppKit call is interpreted by the Display PostScript system as a request to synchronize the NeXTdimension, PostScript interpreter, and app, and will not return until PostScript rendering is complete on the NeXTdimension.
\par
%Running on a color capable machine, the GUI was the most beautiful interface of its time.\\ 
\vspace{5mm} \label{nextstep_color.png}
% \scaledimage{0.95}{nextstep_color.png} 
\cfullimage{nextstep_color.png}{NeXTSTEP running on system with a 24-bit NeXTdimension}
\par
Figure \ref{nextstep_color.png} shows NeXTSTEP running on a 24-bit color machine. The composition was more gorgeous than what any of its competitors was capable of. Notice \cw{Mail.app}, which shipped with an email from Steve Jobs lauding the merits of object-oriented programming.

\section{NeXT at id Software}
To say NeXT polarized professionals would be an understatement. It is fair to say that everybody had a strong opinion about them. Some developers hated it.\\
\par
\fq{Develop for it? I'll piss on it!}{Bill Gates\footnotemark.}\\
\par
Some were interested in having access to a stable Unix system with a powerful GUI.\\
\par
\footnotetext{Found in Walter Isaacson's book "Steve Jobs" but with no source. Bill Gates may have never said that.}
\fq{
When id Software was stationed in Madison, Wisconsin during the winter of 1991, most of us were gone for the Christmas holiday - except John Carmack. John's present, which he bought with \$11,000 of his own money, procured by walking through the snow and ice to remove from the bank\footnotemark, arrived during the holiday and he spent the whole time learning as much as he could about the computer and started working on vector quantization algorithms for compressing graphics. His test graphic was a 256-color screen from King's Quest 5.\\
\par

 After his research was done it was agreed that the entire company needed to develop our next game on NeXTSTEP.}{John Romero, rome.ro, December 20, 2006}\\
 \footnotetext{Emptying his bank account in the process.}
\par
Given the timing of his purchase, the NeXT was first used to produce the Wolfenstein 3D hint book. One of the best DTP applications at the time, \cw{FrameMaker.app} proved perfect for the task. After that, NeXT rapidly conquered id Software for all other operations. PCs remained in use mostly for Deluxe Paint and to test the game they wanted to ship.\\
\par
As the needs of the studio evolved with more and more power, id took advantage of NeXTSTEP's ability to run on various platforms from HP and Intel (called the "White hardware").






The studio bought so much hardware over the years that twenty-five years later, accounts differ about which was the first kind of NeXT machine purchased. According to John Carmack it was a NeXTstation.\\
\par
\vspace{2mm}

\fq{
I bought our first NeXT (a ColorStation) just out of personal interest. Jason Blochowiak had talked to me about the advantages of Unix based systems from his time at college, and I was interested in seeing what Steve Job's next big thing was. It is funny to look back - I can remember honestly wondering what the advantages of a real multi process development environment would be over the DOS and older Apple environments we were using. I remember saying "What else would you do when the compiler was running?".  Jason was ahead of the game when we first met; he was using an expensive Mac II system to cross develop for the lower end Apple IIGS.  He was of course proven right on the value proposition. Actually using the NeXT was an eye opener, and it was quickly clear to me that it had a lot of tangible advantages for us, so we moved everything but pixel art (which was still done in Deluxe Paint on DOS) over. Using Interface Builder for our game editors was a NeXT unique advantage, but most Unix systems would have provided similar general purpose software development advantages (the debugger wasn't nearly as good as Turbo Debugger 386, though!). Kevin Cloud even did our game manuals, starting with Wolfenstein 3D, in Framemaker on a NeXT.\\
\par
This was all in the context of DOS or Windows 3.x; it was revelatory to have a computer system that didn't crash all the time. By the time Quake 2 came around, Windows NT was in a similar didn't-crash-all-the-time state, it had hardware accelerated OpenGL, and Visual Studio was getting really good, so I didn't feel too bad moving over to it. At that transition point I did evaluate most of the other Unix workstations, and didn't find a strong enough reason not to go with Microsoft for our desktop systems.\\
\par
Over the entire course of Doom and Quake 1's development we probably spent \$100,000 on NeXT computers, which isn't much at all in the larger scheme of development. We later spent more than that on Unix SMP server systems (first a quad Alpha, then an eventually 16-way SGI system) to run the time consuming lighting and visibility calculations for the Quake series. I remember one year looking at the Top 500 supercomputer list and thinking that if we had expanded our SGI to 32 processors, we would have just snuck in at the bottom.}{John Carmack, quora.com}\\
\par
John Romero remembers first buying a monochrome NeXTcube.\\
\par
\fq{The NeXTCube was purchased in December 1991 and was the only NeXT Computer we had until December 1992 when we decided we would develop DOOM with them so we bought 3 NeXTStations: mine, John Carmack's new one, Tom Hall's. John C's original NeXTCube was the computer used to scan the clay models, gun toys, and latex models.\\
\par
id's first NeXT hardware was all black - both Cubes and Stations. We upgraded through the years to the Turbo model then to other hardware like the HP Gecko and then Intel hardware at the end. We were building fat binaries of the tools for all 3 processors in the office - one .app file that had code for all 3 processors in it and executed the right code depending on which machine you ran it on. All our data was stored on a Novell 3.11 server and we constantly used the NeXTSTEP Novell gateway object to transparently copy our files to and from the server as if it was a local NTFS drive. This was back in 1993!\\
\par
In fact, with the superpower of NeXTSTEP, one of the earliest incarnations of DoomEd had Carmack in his office, me in my office, DoomEd running on both our computers and both of us editing one map together at the same time. I could see John moving entities around on my screen as I drew new walls. Shared memory spaces and distributed objects. Pure magic.\\}{John Romero}\\
\par
John Romero in particular liked their production pipeline so much that he decided to champion it. He managed to successfully advertise it to another gaming studio.\\
\par
\fq{
DOOM, DOOM II and Quake weren't the only games developed on NeXTSTEP. When I got Raven Software to agree to develop Heretic for us I had them buy several Epson NeXT computers (Intel based) and I flew up to Madison, WI to get them all set up and teach them how to develop the game with our tools and engine. It was a great time I'll never forget - seeing their team get excited about the power of the new environment and that they got the game developed and released in under a year. They signed on for another title and developed Hexen on NeXTSTEP as well.}{John Romero, rome.ro, December 20, 2006}.











\section{Roller coaster}
During its seven years in business, NeXT lead a tumultuous life. It was sued by Apple within its first month of existence (the five people Steve Jobs had taken with him were not "minor people" as he had told Apple). Carried on by Steve Jobs' "reality distortion field", the company was praised for several years even though it had yet to produce anything. Glorified upon each release, owing a lot to marketing genius, the machines later struggled to find customers. Disappointing sales led to a near death experience before NeXT managed to re-invent itself.\\
\par
\vspace{-5pt}
\subsection{Downfall}
As elegant and powerful as the black hardware was, their high price tag made them a dealbreaker. Even the "cheaper" NeXTstations were well beyond most developers' budgets.\\
\par
 In 1988, the factory was building 400 units/month, well below its maximum 10,000/month capacity. Sales worsened in 1989 with only 360 units/month sold over the year. Production had to be slowed down to 100 units/month to avoid overflowing storage. By 1990 things improved slightly with \$28 million in revenue -- still a far cry from the \$2.8 billions Sun Microsystems generated that same year. 1991 saw yet another improvement with 20,000 units sold and a revenue of \$127 million. That figure was still less than what Apple sold in a single week. By 1992, sales reached \$140 million\footnote{Source:The Next Big Thing.} and NeXT claimed its first profitable quarter\footnote{In fact it was \$40 million from breaking even. Source: "The Next Big Thing"}, seven years after its founding.\\
\par
Despite the steady improvements, NeXT still lost \$40 million that year. With only 50,000 NeXT machines sold over the course of its short life\footnote{In 1993 Apple sold 50,000 units every six days.} (including thousands to the then secret National US Reconnaissance Office), Jobs decided that NeXT could not carry on as a hardware manufacturer. Struggling to close deals and hemorrhaging cash, it fired 300 out of its 500 employee workforce on a Black Tuesday of February 1993 to become a software-only company.\\
\par
 The purpose of NeXTSTEP was changed. From operating system in charge of making the black hardware sing, it was to become a white hardware enabler and the sole money maker. It was re-factored to be portable and capable of running on Intel, SPARC, and PA-RISC CPUs. Sold as a combination operating system and object-oriented development environment. NeXTSTEP for Intel became a popular product among large companies and especially financial institutions for rapidly developing and deploying custom software. NeXT also started to collaborate with Sun Microsystems on a light version of NeXTSTEP called OpenSTEP.\\
Even in its darkest hours, NeXT curiously never capitalized on id Software's appreciation for the platform\footnote{comp.sys.next.advocacy: "DOOM: NeXTstep's Most Successful App".}. Reportedly due to Steve Jobs' disdain for video games, they even turned down an opportunity which could have helped them tremendously.\\
\par

\fq{We loved our NeXTs, and we wanted to launch Doom with an explicit "Developed on NeXT computers" logo during the startup process\protect\footnotemark, but when we asked, the request was denied.}{John Carmack}\\
\footnotetext{Tom Hall's "DOOM Bible" mentioned designing maps with labs featuring "a lot of NeXT-looking computers".}
\par
\cscaledimage{1}{credits/DoomCREDIT10.png}{\doom{} alpha version credit screen featuring "NeXT Computers"}
\par

 % id Software also wanted to mention the machines in the credit screen. "DEVELOPED ON NeXT COMPUTERS" proudly stood there until NeXT refusal to be associated with \doom. 
 As \doom{}'s success grew and became a world-wide hit, NeXT backtracked and attempted to reverse its decision but by then, as recalls John Carmack, "that ship had sailed"\footnote{"I showed up for him" by John Carmack. \cw{facebook.com} May 14th, 2018.}.


\subsection{Rebirth}
The rest of \NeXTns{}'s history is the envy of a Hollywood plot-twist. In 1995, Apple Computer's failed attempt at producing its own operating system under project "Copland" had placed the company in a precarious situation with nothing to replace its outdated System 7.\\
\par
 After briefly considering BeOS, Apple elected to buy NeXT in 1996 for \$429 million in cash. The timing could not have been better for \NeXT which was on the verge of bankruptcy. Steve Jobs returned to the company he had co-founded twenty years earlier. After the sale, he first worked as an advisor but was later appointed acting-CEO, to finally become CEO of the company. This was not only a rebirth for \NeXTns, it was also a rebirth for Apple, which went from being ninety days away from insolvency\footnote{Source: "All Things Digital conference, Jun 1, 2010."} in 1996 to most valuable company in the world in December 2017. \\
\par
NeXTSTEP was used as foundation for Apple's Rhapsody project which became Darwin, the core of Apple's OSes. The new operating system was met with enthusiasm by customers and professionals who praised the design, look and feel, and stability. Darwin was later used as a base for Apple's 2008 iPhone which took the world by storm. It is extremely likely some of the code that once ran on NeXTSTEP and contributed to \doom{} now runs on the many millions of Apple computers and iOS phones across the world.\\



